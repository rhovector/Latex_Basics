%%%%%%%%%%%%%%%%%%%%%%%%%%%%%%%%%%%%%%%%%%%%%%%%%%%%%%%%%%%%%%%%%%%%%%%%%%%%%%%%%%
%%      Template using LaTeX  Tutorial tex files								%%
%%      																		%%
%%      Copyright Rho Vector Latex Tutorial 2021								%%
%%		rhovector <rhovector@gmail.com>			 								%%
%%		vivekadi <vivek.adishesha@gmail.com>,									%%
%%		chiranjitpatel <chiranjitpatel08@gmail.com>								%%
%%																				%%
%%																				%%	
%%      This program is FREE SOFTWARE; you can redistribute it and/or modify	%%
%%      it under the terms of the GNU General Public License as published by	%%
%%      the Free Software Foundation; either version 2 of the License, or		%%
%%      (at your option) any later version.										%%
%%      																		%%
%%      This program is distributed in the hope that it will be useful,			%%
%%      but WITHOUT ANY WARRANTY; without even the implied warranty of			%%
%%      MERCHANTABILITY or FITNESS FOR A PARTICULAR PURPOSE.					%%
%%      See GNU General Public License for more details.						%%
%%      																		%%
%%      You should have received a copy of the GNU General Public License		%%
%%      along with this program; if not, write to the Free Software				%%
%%      Foundation, Inc., 51 Franklin Street, Fifth Floor, Boston,				%%
%%      MA 02110-1301, USA.														%%
%%%%%%%%%%%%%%%%%%%%%%%%%%%%%%%%%%%%%%%%%%%%%%%%%%%%%%%%%%%%%%%%%%%%%%%%%%%%%%%%%%




\documentclass{article}
\usepackage{xcolor}

%% Package "listings" for code snippets inclusion to the document %%
\usepackage{listings}

%% Backgorund and other color combinations in code snippets %%
\definecolor{codegreen}{rgb}{0,0.6,0}
\definecolor{codegray}{rgb}{0.5,0.5,0.5}
\definecolor{codepurple}{rgb}{0.58,0,0.82}
\definecolor{backcolour}{rgb}{1,1,1}

\lstdefinestyle{mystyle}{
	backgroundcolor=\color{backcolour},   
	commentstyle=\color{codegreen},
	keywordstyle=\color{magenta},
	numberstyle=\tiny\color{codegray},
	stringstyle=\color{codepurple},
	basicstyle=\ttfamily\footnotesize,
	breakatwhitespace=false,         
	breaklines=true,                 
	captionpos=b,                    
	keepspaces=true,                 
	numbers=left,                    
	numbersep=5pt,                  
	showspaces=false,                
	showstringspaces=false,
	showtabs=false,                  
	tabsize=2
}

\lstset{style=mystyle}
%% end color for code snippet %%


\begin{document}
	
	%% The verbatim environment %%
	\begin{verbatim}
		Text enclosed inside \texttt{verbatim} environment 
		is printed directly 
		and all \LaTeX{} commands are ignored.
	\end{verbatim}
	
	Just as in the example at the introduction, all text is printed keeping line breaks and white spaces. There's a starred version of this command whose output is slightly different.
	\begin{verbatim*}
		Text enclosed inside \texttt{verbatim} environment 
		is printed directly 
		and all \LaTeX{} commands are ignored.
	\end{verbatim*}
	
	=========================================\\\\
	%%%%%%%%%%%%%%%%%%%%%%%%%%%%%%%%%%%%%%%%%%%%%%%%%%%%%%%%%%%%%%%%%%%%%%%%%
	
	%% Example 2 - Using listings to highlight code %%
	To use the lstlisting environment you have to add the following line to the preamble of your document:
	
	\textbackslash usepackage\{listings\}
	
	\begin{lstlisting}
		name = input('What is your name?\n')
		print ('Hi, %s.' % name)
	\end{lstlisting}
	
	In this example, the output ignores all LATEX commands and the text is printed keeping all the line breaks and white spaces typed. Let's see a second example:\\
	
	\begin{lstlisting}[language=Python]
		name = input('What is your name?\n')
		print ('Hi, %s.' % name)
	\end{lstlisting}
	The additional parameter inside brackets [language=Python] enables code highlighting for this particular programming language (Python), special words are in boldface font and comments are italicized.\\
	
	Below is an example of adding a C++ snippet\\
	\begin{lstlisting}[language=C++]
		#include <iostream>
		using namespace std;
		
		// main() is where program execution begins.
		int main() {
			cout << "Hello World"; // prints Hello World
			return 0;
		}
	\end{lstlisting}
	
	
	==========================================\\\\
	%%%%%%%%%%%%%%%%%%%%%%%%%%%%%%%%%%%%%%%%%%%%%%%%%%%%%%%%%%%%%%%%%%%%%%%%%
	
	%% Example 3 - Importing code from a file %%
	Code is usually stored in a source file, therefore a command that automatically pulls code from a file becomes very handy.\\
	
	The next code will be directly imported from a file
	
	\lstinputlisting[language=Octave]{Area_of_circle.txt}
	As you see, the code colouring and styling greatly improves readability.\\
	There are essentially two commands that generate the style for this example:\\
	
	\textbackslash lstdefinestyle\{mystyle\}\{...\}\\
	Defines a new code listing style called "mystyle". Inside the second pair of braces the options that define this style are passed; see the reference guide for a full description of these and some other parameters.\\
	\textbackslash lstset\{style=mystyle\}\\
	Enables the style "mystyle". This command can be used within your document to switch to a different style if needed.\\
	
	==========================================\\\\
	%%%%%%%%%%%%%%%%%%%%%%%%%%%%%%%%%%%%%%%%%%%%%%%%%%%%%%%%%%%%%%%%%%%%%%%%%
	
	%% Example 4 - Captions and the list of Listings %%
	\begin{lstlisting}[language=Python, caption= Python Program Example]
		name = input('What is your name?\n')
		print ('Hi, %s.' % name)
	\end{lstlisting}
	
	Adding the comma-separated parameter caption=Python example inside the brackets, enables the caption. This caption can be later used in the list of Listings.\\
	
	\textbackslash lstlistoflistings\\
	\lstlistoflistings
	
	==========================================\\\\
	%%%%%%%%%%%%%%%%%%%%%%%%%%%%%%%%%%%%%%%%%%%%%%%%%%%%%%%%%%%%%%%%%%%%%%%%%
	
	%% Notes %%
	Options to customize code listing styles\\
	backgroundcolor - colour for the background. External color or xcolor package needed.\\
	commentstyle - style of comments in source language.\\
	basicstyle - font size/family/etc. for source (e.g. basicstyle=\ttfamily\small)\\
	keywordstyle - style of keywords in source language (e.g. keywordstyle=\textbackslash color\{red\})\\
	numberstyle - style used for line-numbers\\
	numbersep - distance of line-numbers from the code\\
	stringstyle - style of strings in source language\\
	showspaces - emphasize spaces in code (true/false)\\
	showstringspaces - emphasize spaces in strings (true/false)\\
	showtabs - emphasize tabulators in code (true/false)\\
	numbers - position of line numbers (left/right/none, i.e. no line numbers)\\
	prebreak - displaying mark on the end of breaking line (e.g. prebreak=\textbackslash raisebox\{0ex\}[0ex][0ex]\{\textbackslash ensuremath\{\textbackslash hookleftarrow\}\})\\
	captionpos - position of caption (t/b)\\
	frame - showing frame outside code (none/leftline/topline/bottomline/lines/single/shadowbox)\\
	breakwhitespace - sets if automatic breaks should only happen at whitespaces
	breaklines - automatic line-breaking
	keepspaces - keep spaces in the code, useful for indetation
	tabsize - default tabsize
	rulecolor - Specify the colour of the frame-box
	
	==========================================\\\\
	%%%%%%%%%%%%%%%%%%%%%%%%%%%%%%%%%%%%%%%%%%%%%%%%%%%%%%%%%%%%%%%%%%%%%%%%%
	
	%% Example 5 - Python code %% 
	\textbf{\textit{Example Python Code}}
	\begin{lstlisting}[language=python]
		import qiskit as q
		from qiskit import Aer,execute
		from qiskit import IBMQ
		from qiskit.tools.visualization import plot_histogram,plot_bloch_multivector
		from qiskit.tools.monitor import job_monitor
		import matplotlib
		statevector=q.Aer.get_backend("statevector_simulator")#statevector simulator
		qasm_sim=q.Aer.get_backend("qasm_simulator")#Quantum simulator
		def do_jobs(circuit):
		result=execute(circuit,backend=statevector).result()#statevectors of statevec 
		statevec = result.get_statevector()
		n_qubits=circuit.num_qubits#get total no of qubits in circuit
		circuit.measure([i for i in range(n_qubits)],[i for i in range(n_qubits)])# measure qubits and store in classical bits
		
		result2=execute(circuit,backend=qasm_sim).result()#qasm simulator
		counts = result2.get_counts()
		return statevec,counts
		
		circuit=q.QuantumCircuit(2,2)#qubit=2 and classical=2
		circuit.h(0)#H gate on 1st qubit 
		circuit.x(1)#X gate on 2nd qubit
		circuit.h(1)#X gate on 2nd qubit
		statevec,counts=do_jobs(circuit)
		circuit.draw(output="mpl",filename='superpositionckt.png')
		plot_bloch_multivector(statevec).show()#plot blochsphere
		plot_histogram(counts).show()#plot histogram
	\end{lstlisting}
	==========================================
	%%%%%%%%%%%%%%%%%%%%%%%%%%%%%%%%%%%%%%%%%%%%%%%%%%%%%%%%%%%%%%%%%%%%%%%%%
	
\end{document}